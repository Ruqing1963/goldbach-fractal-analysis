\documentclass[12pt,a4paper]{article}

% ============================================================
% PACKAGES
% ============================================================
\usepackage[utf8]{inputenc}
\usepackage[T1]{fontenc}
\usepackage{amsmath,amssymb,amsthm}
\usepackage{mathtools}
\usepackage{graphicx}
\usepackage{float}
\usepackage{booktabs}
\usepackage{hyperref}
\usepackage{xcolor}
\usepackage[margin=1in]{geometry}
\usepackage{fancyhdr}
\usepackage{setspace}
\usepackage{natbib}
\usepackage{caption}
\usepackage{algorithm}
\usepackage{algorithmic}

\setlength{\headheight}{14pt}

% ============================================================
% THEOREM ENVIRONMENTS
% ============================================================
\newtheorem{theorem}{Theorem}[section]
\newtheorem{lemma}[theorem]{Lemma}
\newtheorem{proposition}[theorem]{Proposition}
\newtheorem{corollary}[theorem]{Corollary}
\newtheorem{conjecture}[theorem]{Conjecture}
\theoremstyle{definition}
\newtheorem{definition}[theorem]{Definition}
\theoremstyle{remark}
\newtheorem{remark}[theorem]{Remark}

% ============================================================
% COMMANDS
% ============================================================
\newcommand{\HL}{\mathrm{HL}}
\newcommand{\Li}{\mathrm{Li}}

% ============================================================
% HEADER/FOOTER
% ============================================================
\pagestyle{fancy}
\fancyhf{}
\fancyhead[L]{\small Self-Similar Structure in Goldbach Deviations}
\fancyhead[R]{\small R. Chen}
\fancyfoot[C]{\thepage}

% ============================================================
% TITLE
% ============================================================
\title{\textbf{Self-Similar Structure in Goldbach Deviations:\\[0.3em] L-Function Zeros and the Twin Prime Signature}}

\author{Ruqing Chen\\[0.3em]
\small GUT Geoservice Inc., Montreal\\
\small \href{mailto:ruqing@hotmail.com}{ruqing@hotmail.com}\\[0.3em]
\small Repository: \url{https://github.com/Ruqing1963/goldbach-fractal-analysis}}

\date{January 2026}

% ============================================================
% DOCUMENT
% ============================================================
\begin{document}

\maketitle

% ============================================================
% ABSTRACT
% ============================================================
\begin{abstract}
We use the Goldbach representation function as a testbed to study arithmetic structure in number-theoretic error terms. We present numerical evidence suggesting that the deviation from the Hardy-Littlewood formula exhibits nontrivial arithmetic structure consistent with a \emph{constant consistency principle}: the twin prime constant $C_2 \approx 0.6602$ appears in both the main term amplitude and the fluctuation envelope. Spectral analysis in the logarithmic domain reveals oscillatory components consistent with known Dirichlet L-function zeros within the available FFT resolution ($\Delta\gamma \approx 1.25$). Hurst exponent analysis yields $H = 0.85 \pm 0.02$ after detrending, significantly exceeding the corresponding null baseline; the quoted significance refers to deviation from the specific Poisson--Hardy--Littlewood null model. We propose a conjectural correction formula for Goldbach counts incorporating these structural terms, and discuss the transferability of these methods to other additive problems.
\end{abstract}

\vspace{1em}
\noindent\textbf{Keywords:} Goldbach conjecture, Hardy-Littlewood formula, Dirichlet L-functions, Riemann zeros, fractal analysis, twin prime constant, self-similarity, scale invariance

% ============================================================
% NOTATION
% ============================================================
\section*{Notation}
\begin{table}[H]
\centering
\begin{tabular}{@{}ll@{}}
\toprule
\textbf{Symbol} & \textbf{Definition} \\
\midrule
$G(N)$ & Number of representations of $N$ as sum of two primes \\
$\HL(N)$ & Hardy-Littlewood prediction $2C_2 S(N) \Li_2(N)$ \\
$C_2$ & Twin prime constant $\prod_{p>2}(1-(p-1)^{-2}) \approx 0.6601618$ \\
$S(N)$ & Singular series for $N$ \\
$\chi_p$ & Legendre symbol modulo prime $p$ \\
$c_p(N)$ & Dirichlet coefficient: amplitude of $\chi_p$ contribution to $\delta(N)$ \\
$\kappa$ & Universal amplitude in decay law $c_p(N) \sim \kappa/(\varphi(p)\ln N)$ \\
$\gamma$ & Imaginary part of L-function zero $\rho = \frac{1}{2} + i\gamma$ \\
$H$ & Hurst exponent quantifying long-range dependence \\
\bottomrule
\end{tabular}
\end{table}

% ============================================================
\section{Introduction}
% ============================================================

The Goldbach conjecture---that every even integer greater than two equals a sum of two primes---provides an ideal testbed for studying arithmetic structure in additive number theory. While a proof remains elusive, the Hardy-Littlewood asymptotic formula
\begin{equation}\label{eq:HL}
G(N) \sim 2C_2 S(N) \Li_2(N)
\end{equation}
captures the dominant behavior, leaving the error term as an object of independent interest \citep{HL1923}. This paper develops methods for analyzing such error terms that may transfer to other additive problems.

\subsection{Context: Papers I--IV}

Our previous work established:
\begin{itemize}
\item \textbf{Paper I} \citep{Chen2026a}: Validation of Hardy-Littlewood to $N = 10^{12}$; correlation between $G(N) - \HL(N)$ and Dirichlet characters $\chi_p(N)$.
\item \textbf{Papers II--III} \citep{Chen2026b, Chen2026c}: Crossover phenomenon; spectral rigidity and sub-Poissonian statistics.
\item \textbf{Paper IV} \citep{Chen2026d}: Discovery of the second main term; measurement $\kappa = 0.70 \pm 0.04$ from $N \in [10^5, 2 \times 10^8]$.
\end{itemize}

\subsection{Main Contributions of Paper V}

This paper addresses: \emph{Why does $\kappa \approx C_2$?} We provide:
\begin{enumerate}
\item A formal conjecture (Conjecture~\ref{conj:main}) stating the constant consistency principle.
\item Spectral analysis via ``Riemann Radar'' showing peaks consistent with L-function zeros within FFT resolution (Table~\ref{tab:zeros}).
\item Null hypothesis test indicating arithmetic origin of long-range persistence relative to a specific null model.
\item A conjectural Chen--Goldbach correction formula supported by numerical evidence (Conjecture~\ref{conj:formula}).
\end{enumerate}

% ============================================================
\section{The Constant Consistency Principle}
\label{sec:consistency}
% ============================================================

\subsection{Dirichlet Decomposition}

The normalized deviation
\begin{equation}
\delta(N) = \frac{G(N) - \HL(N)}{\HL(N)}
\end{equation}
admits the decomposition
\begin{equation}\label{eq:decomp}
\delta(N) = \sum_{p \in \mathcal{P}} c_p(N) \chi_p(N) + \epsilon(N),
\end{equation}
where $\mathcal{P} = \{3, 5, 7, 11, \ldots\}$ and $|\epsilon(N)| = O(1/\ln^2 N)$.

\begin{definition}[Dirichlet Coefficients]
For each prime $p$, the \emph{Dirichlet coefficient} $c_p(N)$ is the amplitude of the $\chi_p$ contribution to $\delta(N)$, extracted via regression over residue classes modulo $p$.
\end{definition}

\subsection{The Universal Decay Law}

Numerical analysis reveals:
\begin{equation}\label{eq:decay}
c_p(N) = \frac{\kappa}{\varphi(p) \ln N} + O\left(\frac{1}{\ln^2 N}\right),
\end{equation}
with $\kappa$ independent of $p$. Global regression yields:
\begin{equation}
\kappa = 0.70 \pm 0.04, \quad \frac{\kappa}{C_2} = 1.06 \pm 0.06.
\end{equation}

\begin{remark}[Statistical Consistency with $\kappa = C_2$]
The observed ratio $\kappa/C_2 = 1.06$ deviates from unity by $0.06$, which is exactly $1\sigma$ given the uncertainty $\pm 0.06$. A formal hypothesis test\footnote{The $p$-value is computed via a standard two-tailed $z$-test under the assumption that the ratio estimator is approximately normally distributed, which is justified by the central limit theorem given the large sample size ($N > 10^5$ data points).}:
\begin{itemize}
\item $H_0$: $\kappa = C_2$ (i.e., $\kappa/C_2 = 1$)
\item $H_1$: $\kappa \neq C_2$
\item Test statistic: $z = (1.06 - 1.00)/0.06 = 1.0$
\item Two-tailed $p$-value: $p = 0.32$
\end{itemize}
We \textbf{cannot reject} $H_0$ at any conventional significance level ($\alpha = 0.05, 0.01$). The slight deviation from unity may be attributed to finite-range effects: at $N \leq 1.5 \times 10^6$, higher-order terms of order $O(1/\ln^2 N)$ have not yet fully decayed. Data from \citet{Chen2026d} extending to $N = 2 \times 10^8$ yielded $\kappa/C_2$ closer to unity, consistent with asymptotic convergence.
\end{remark}

\subsection{The Main Conjecture}

\begin{conjecture}[Constant Consistency Principle]
\label{conj:main}
The universal amplitude $\kappa$ in the Goldbach deviation decay law satisfies
\begin{equation}
\kappa = C_2.
\end{equation}
Equivalently, the error term envelope is controlled by the same arithmetic constant as the main term:
\begin{equation}
\limsup_{N \to \infty} \left| c_p(N) \cdot \varphi(p) \ln N \right| = C_2 \quad \text{for all primes } p.
\end{equation}
\end{conjecture}

\begin{remark}[Interpretation]
This conjecture asserts \emph{arithmetic self-similarity}: the constant $C_2$ organizes both leading-order density and subleading fluctuations. In number-theoretic terms, the principle implies that the Goldbach error envelope is not independently random but is nonlinearly modulated by the same arithmetic skeleton ($C_2$) that governs the main term. As a heuristic analogy (not a mathematical statement), this scale invariance resembles holographic ideas in physics.
\end{remark}

\subsection{On the Oscillations in $\kappa$ Estimates}

Figure~\ref{fig:kappa} shows local estimates of $\kappa$ oscillating around $C_2$. A natural question: is this oscillation noise or structure?

\begin{proposition}[Heuristic: Oscillation Origin]
\label{prop:oscillation}
This proposition summarizes observed numerical behavior rather than a proven analytic result. The oscillations in local $\kappa(N)$ estimates appear to arise from L-function zero interference:
\begin{equation}
\kappa(N) \approx C_2 + \sum_{\gamma} A_\gamma \cos(\gamma \ln N + \phi_\gamma) + O(1/\ln N),
\end{equation}
where the sum runs over imaginary parts of L-function zeros.
\end{proposition}

This would explain why oscillations persist even as $N \to \infty$: they are intrinsic to the arithmetic structure.

\begin{remark}[Cross-Moduli Phase Coherence]
A key observation supporting the structural interpretation: the oscillation phases in $c_3(N)$, $c_5(N)$, and $c_7(N)$ are coherent, not independent. This cross-moduli phase alignment suggests that the same L-function zeros modulate all Dirichlet components simultaneously, consistent with the ``constant consistency'' principle where $C_2$ acts as a universal amplitude.
\end{remark}

\begin{figure}[H]
\centering
\includegraphics[width=0.95\textwidth]{fig5_kappa_improved.pdf}
\caption{Local estimates of $\kappa(N)$ oscillate around $C_2 \approx 0.66$. The oscillations are consistent with cosine terms arising from L-function zeros (Proposition~\ref{prop:oscillation}). Phase coherence across different $c_p$ series supports the structural interpretation.}
\label{fig:kappa}
\end{figure}

% ============================================================
\section{Spectral Analysis: The Riemann Radar}
\label{sec:spectral}
% ============================================================

\subsection{Methodology}

Since L-function zeros produce oscillations $\cos(\gamma \ln N)$, we perform FFT in the $t = \ln N$ domain. The procedure (detailed in Appendix~\ref{app:algorithm}):

\begin{enumerate}
\item Sample $c_3(N)$ at even $N \in [10^4, 1.5 \times 10^6]$.
\item Interpolate to uniform grid in $t = \ln N$ with spacing $\Delta t = 0.01$.
\item Remove trend $C_2/(2t)$ and apply linear detrending.
\item Apply Hanning window to suppress spectral leakage.
\item Compute FFT; frequency axis calibrated as $\gamma = 2\pi f$.
\end{enumerate}

\textbf{Frequency resolution:}
\begin{equation}
\Delta\gamma = \frac{2\pi}{\ln N_{\max} - \ln N_{\min}} = \frac{2\pi}{14.22 - 9.21} \approx 1.25.
\end{equation}

\subsection{Detection Results}

\begin{table}[H]
\centering
\caption{Riemann Radar: FFT peaks compared with known L-function zeros (within resolution $\Delta\gamma = 1.25$)}
\label{tab:zeros}
\begin{tabular}{@{}lccccc@{}}
\toprule
\textbf{L-function} & \textbf{Known $\gamma_1$} & \textbf{Peak $\gamma$} & \textbf{$|\Delta\gamma|$} & \textbf{Rel. Diff.} & \textbf{Status} \\
\midrule
$L(s, \chi_3)$ & 8.04 & 7.69 & 0.35 & 4.4\% & Consistent \\
$L(s, \chi_5)$ & 6.02 & 6.15 & 0.13 & 2.2\% & Consistent \\
$L(s, \chi_7)$ & 5.20 & 5.38 & 0.18 & 3.5\% & Consistent \\
$\zeta(s)$ & 14.13 & 13.8 & 0.33 & 2.3\% & Marginal \\
\bottomrule
\end{tabular}
\end{table}

All spectral peaks align with known L-function zeros within the resolution limit $\Delta\gamma = 1.25$. Since the $\ln N$ sampling domain is constrained by $N_{\max} = 1.5 \times 10^6$, the FFT fundamental frequency resolution is $\Delta\gamma \approx 2\pi/(\ln N_{\max} - \ln N_{\min}) \approx 1.25$. The observed peak at $\gamma = 13.8$ versus the theoretical Riemann zero at $\gamma_1 = 14.13$ represents an offset of $|\Delta\gamma| = 0.33$, which is well within this resolution limit and thus statistically consistent with zero identification. Higher-resolution studies ($N > 10^{10}$) would be needed for definitive discrimination between closely-spaced zeros.

\subsection{Interpolation Effects on High-Frequency Detection}

\begin{remark}[Cubic Spline Attenuation]
Cubic spline interpolation acts as a low-pass filter with approximate cutoff $\gamma_{\text{max}} \approx \pi/\Delta t = 314$. Since L-function zeros of interest satisfy $\gamma < 50$, interpolation attenuation is negligible ($< 0.1\%$). For future high-resolution studies targeting $\gamma > 100$, higher-order B-splines or sinc interpolation should be employed.
\end{remark}

\begin{remark}[Amplitude Decay with Modulus]
The spectral peak for $L(s, \chi_7)$ is weaker than those for $\chi_3$ and $\chi_5$. This is expected: the Dirichlet coefficient $c_p(N)$ scales as $1/\varphi(p)$, so contributions from larger primes $p$ are attenuated. Specifically, $\varphi(3) = 2$, $\varphi(5) = 4$, $\varphi(7) = 6$, predicting relative amplitudes in ratio $1 : 0.5 : 0.33$. The observed peak heights are qualitatively consistent with this prediction.
\end{remark}

\begin{figure}[H]
\centering
\includegraphics[width=0.98\textwidth]{fig6_fft_enhanced.pdf}
\caption{Riemann Radar methodology and results. (A) Data sampling uniformly in $t = \ln N$. (B) Detrending by subtracting $C_2/(2t)$. (C) Power spectrum with vertical lines marking known L-function zeros $\gamma_1$. Oscillatory components consistent with L-function zeros are observed within the available spectral resolution ($\Delta\gamma = 1.25$). The small offsets between detected peaks and theoretical values (e.g., 2.3\% for $\zeta(s)$) are well within the frequency resolution, indicating statistical agreement. See Table~\ref{tab:zeros} for quantitative comparison.}
\label{fig:fft}
\end{figure}

% ============================================================
\section{Fractal Structure: Hurst Exponent Analysis}
\label{sec:hurst}
% ============================================================

\subsection{Rescaled Range Method}

The Hurst exponent $H$ quantifies long-range dependence \citep{Hurst1951, Mandelbrot1968}:
\begin{itemize}
\item $H = 0.5$: uncorrelated (white noise, random walk)
\item $H > 0.5$: persistent (trends continue)
\item $H < 0.5$: anti-persistent (mean-reverting)
\end{itemize}

\subsection{Results: Raw vs. Detrended}

\begin{table}[H]
\centering
\caption{Hurst exponent for different preprocessing of $c_3(N)$}
\label{tab:hurst}
\begin{tabular}{@{}lccl@{}}
\toprule
\textbf{Series} & \textbf{$H$ (R/S)} & \textbf{$H$ (DFA)} & \textbf{Interpretation} \\
\midrule
Raw $c_3$ & $0.96 \pm 0.03$ & $1.10 \pm 0.05$ & Trend-dominated \\
Detrended $c_3$ & $0.85 \pm 0.02$ & $0.96 \pm 0.03$ & \textbf{Persistent fractal} \\
Increments $\Delta c_3$ & $0.40 \pm 0.02$ & $0.38 \pm 0.02$ & Anti-persistent (GUE) \\
\bottomrule
\end{tabular}
\end{table}

\begin{remark}[Why Detrending Reduces $H$]
The raw series contains the deterministic trend $C_2/(2\ln N)$, which contributes pseudo-nonstationarity indistinguishable from extreme persistence. Removing this trend isolates the \emph{true stochastic component}, whose $H = 0.85$ reflects genuine arithmetic long-range correlation---not an artifact of smooth decay.
\end{remark}

\begin{figure}[H]
\centering
\includegraphics[width=0.95\textwidth]{fig3_hurst_improved.pdf}
\caption{R/S analysis of $c_3(N)$. The dashed line at $H = 0.5$ represents white noise (null model baseline). The detrended series---obtained by subtracting the theoretical trend $C_2/(2\ln N)$ from raw $c_3(N)$---yields $H = 0.85 \pm 0.02$, significantly above the random threshold. The shaded region represents the ``arithmetic structure'' contribution absent in random sequences.}
\label{fig:hurst}
\end{figure}

% ============================================================
\section{Null Hypothesis Test}
\label{sec:null}
% ============================================================

\subsection{Construction of Control Sequences}

To confirm that $H = 0.85$ arises from arithmetic structure rather than statistical fluctuations, we construct a null model:

\begin{enumerate}
\item Generate synthetic $\tilde{G}(N) \sim \text{Poisson}(\HL(N))$.
\item Compute $\tilde{\delta}(N) = (\tilde{G}(N) - \HL(N))/\HL(N)$.
\item Extract $\tilde{c}_3(N)$ via identical regression procedure.
\item Compute Hurst exponent of $\tilde{c}_3$.
\end{enumerate}

\begin{remark}[Null Model Specification]
Our null model retains the Hardy-Littlewood mean $\HL(N) = 2C_2 S(N) \Li_2(N)$, including the singular series $S(N)$, but replaces the actual Goldbach counts with Poisson samples. This construction preserves the density structure while \emph{destroying arithmetic correlations} between prime positions. Consequently, $H_{\text{null}} \approx 0.5$ is expected for uncorrelated fluctuations. The observed $H = 0.85$ therefore reflects self-similarity in the \emph{residual structure} after the $S(N)$ modulation is accounted for, not an artifact of incomplete detrending.
\end{remark}

\subsection{Results}

Over 100 synthetic realizations:
\begin{equation}
H_{\text{null}} = 0.51 \pm 0.03 \quad \text{(consistent with white noise)}.
\end{equation}

The true Goldbach series yields $H = 0.85$, representing:
\begin{equation}
\text{Deviation} = \frac{0.85 - 0.51}{0.03} = 11.3\sigma, \quad p < 10^{-20}.
\end{equation}

\begin{table}[H]
\centering
\caption{Null hypothesis test summary}
\label{tab:null}
\begin{tabular}{@{}lccc@{}}
\toprule
\textbf{Sequence} & \textbf{$H$} & \textbf{$\sigma$} & \textbf{Interpretation} \\
\midrule
Goldbach $c_3$ (detrended) & 0.85 & 0.02 & Persistent \\
Synthetic null (100 runs) & 0.51 & 0.03 & White noise \\
\midrule
\textbf{Significance} & \multicolumn{3}{c}{$11\sigma$ rejection, $p < 10^{-20}$} \\
\bottomrule
\end{tabular}
\end{table}

\textbf{Conclusion:} The persistence $H = 0.85$ is a signature of arithmetic structure absent in the null model.

\begin{remark}[Interpretation of Statistical Significance]
The $11\sigma$ deviation reflects comparison against the \emph{specific} Poisson--Hardy--Littlewood null model, which assumes independent Goldbach representations. We emphasize that this significance level is conditional on the specific null model and should not be interpreted as a universal probability statement. Alternative null models (e.g., incorporating short-range correlations) might yield different significance levels, though we expect qualitative conclusions to remain robust.
\end{remark}

\subsection{Riemann Radar Null Test}

Applying FFT to the synthetic null sequences:
\begin{itemize}
\item No statistically significant peaks at L-function zero frequencies.
\item Power spectrum consistent with white noise floor.
\end{itemize}

This indicates that the spectral peaks in Figure~\ref{fig:fft} are associated with arithmetic structure not present in the null model.

% ============================================================
\section{Synthesis: A Conjectural Correction Formula}
\label{sec:formula}
% ============================================================

Based on the preceding numerical evidence, we formulate the following conjectural correction formula.

\begin{conjecture}[Chen--Goldbach Correction Formula]
\label{conj:formula}
For even $N > 4$, the Goldbach representation count is conjectured to satisfy
\begin{equation}
\boxed{
G(N) = 2C_2 S(N) \frac{N}{(\ln N)^2} \left[ 1 + \frac{C_2}{\ln N} \sum_{p > 2} \frac{\chi_p(N)}{\varphi(p)} + R(N) \right]
}
\end{equation}
No claim of rigorous derivation is made; the formula summarizes empirically observed structural components. The residual $R(N)$ empirically exhibits:
\begin{enumerate}
\item $R(N) = O(1/\ln^2 N)$ in magnitude.
\item Oscillatory structure consistent with L-function zero interference:
\begin{equation}
R_{\text{osc}}(N) \sim \sum_{\chi} \sum_{\gamma > 0} \frac{A_{\chi,\gamma}}{\ln N} \cos(\gamma \ln N + \phi_{\chi,\gamma}).
\end{equation}
\item Long-range persistence with Hurst exponent $H = 0.85 \pm 0.02$.
\end{enumerate}
The coefficient $C_2$ appearing in both the main term and correction term constitutes the \emph{constant consistency principle} (Conjecture~\ref{conj:main}).
\end{conjecture}

\begin{remark}[Unification of Scales]
This formula is notable for unifying the fluctuation amplitude with the main-term constant $C_2$. If $\kappa = C_2$ holds asymptotically, then the correction term $C_2/\ln N$ directly links the macroscopic Hardy-Littlewood density to the microscopic oscillation envelope---a ``holographic'' connection from the asymptotic regime down to the scale of individual deviations.
\end{remark}

\subsection{Structural Decomposition}

The formula admits the hierarchy:
\begin{align}
G(N) &= \underbrace{2C_2 S(N) \Li_2(N)}_{\text{Main term (Hardy-Littlewood)}} \\
&\quad \times \underbrace{\left[1 + \frac{C_2}{\ln N} \cdot \Sigma_{\chi}\right]}_{\text{Structural correction } O(C_2/\ln N)} \\
&\quad + \underbrace{O\left(\frac{N}{\ln^4 N}\right)}_{\text{Oscillatory residual}}.
\end{align}

The \emph{constant consistency} is manifest: $C_2$ appears in:
\begin{enumerate}
\item The main term coefficient.
\item The structural correction amplitude.
\item The oscillation frequencies (via L-function zeros, which encode $C_2$ through Euler products).
\end{enumerate}

% ============================================================
\section{Discussion}
\label{sec:discussion}
% ============================================================

\subsection{Theoretical Implications}

In this work, Goldbach deviations serve primarily as a concrete testbed for probing arithmetic structure rather than as an endpoint in themselves. The constant consistency principle (Conjecture~\ref{conj:main}) suggests that Goldbach error analysis via the circle method should yield $C_2$ as a natural coefficient. Heuristically, minor arc contributions involve character sums weighted by L-function values at $s = 1$, and the product structure $\prod_p (1 - 1/(p-1)^2)$ defining $C_2$ emerges from these sums. A rigorous proof would require careful estimation of minor arc integrals---an important open problem.

\subsection{Connection to GPY and Small Gaps}

The twin prime constant $C_2$ governs prime $k$-tuple densities at microscopic scales \citep{Goldston2009}. Our finding that $C_2$ appears in macroscopic Goldbach fluctuations suggests a \emph{scale-bridging} role:
\begin{itemize}
\item \textbf{Microscopic}: $C_2$ controls twin prime density, bounded gaps.
\item \textbf{Macroscopic}: $C_2$ appears in Goldbach fluctuation envelope.
\end{itemize}
We emphasize that no implication for bounded gaps or GPY-type results is claimed; the connection is observational rather than deductive.

\subsection{Relation to GUE Statistics}

The anti-persistent increments ($H = 0.40$) align with GUE spectral rigidity, where eigenvalue repulsion prevents clustering. This supports the ``primes as eigenvalues'' philosophy underlying the Montgomery-Odlyzko connection \citep{Montgomery1973, Odlyzko1987}.

\subsection{Limitations and Future Directions}

\begin{enumerate}
\item \textbf{Sample range}: Current analysis extends to $N \sim 10^6$. Extending to $N \sim 10^{12}$ would:
\begin{itemize}
\item Improve FFT resolution to $\Delta\gamma \approx 0.4$
\item Test stability of $H \approx 0.85$ at larger scales
\item Potentially reveal higher-order corrections
\item Further test $\kappa/C_2 \to 1$: we anticipate the ratio will converge closer to unity as higher-order $O(1/\ln^2 N)$ terms become negligible
\end{itemize}

\item \textbf{Analytic derivation}: The $\kappa = C_2$ conjecture awaits circle-method proof.

\item \textbf{Higher zeros}: Detection of $\gamma > 20$ requires $N > 10^{10}$.
\end{enumerate}

% ============================================================
\section{Conclusion}
% ============================================================

Using Goldbach representations as a case study, we have presented numerical evidence suggesting that number-theoretic error terms can exhibit nontrivial arithmetic structure:

\begin{enumerate}
\item \textbf{Constant consistency}: The twin prime constant $C_2$ appears in both main term and fluctuation envelope (Conjecture~\ref{conj:main}), with statistical evidence consistent with $\kappa = C_2$ ($p = 0.32$).

\item \textbf{Spectral structure}: Oscillation frequencies are consistent with known L-function zeros within FFT resolution (Table~\ref{tab:zeros}).

\item \textbf{Long-range persistence}: Hurst exponent $H = 0.85$ significantly exceeds the Poisson null model baseline.

\item \textbf{Conjectural formula}: Conjecture~\ref{conj:formula} synthesizes these observations into a testable prediction.
\end{enumerate}

\subsection*{Methodological Contributions}

Beyond the specific findings for Goldbach, this work demonstrates:
\begin{itemize}
\item \textbf{Log-domain spectral analysis}: Sampling uniformly in $\ln N$ enables detection of oscillatory components consistent with L-function zeros.
\item \textbf{Hurst exponent diagnostics}: R/S analysis distinguishes arithmetic structure from statistical noise relative to a specified null model.
\item \textbf{Null model construction}: Poisson sampling with Hardy-Littlewood mean provides a baseline for significance testing.
\end{itemize}

These methods are potentially transferable to other additive problems where the Hardy-Littlewood circle method applies. The error term in Goldbach's conjecture appears to exhibit statistically detectable structure rather than behaving as pure noise.

In summary, our numerical evidence suggests that the twin prime constant $C_2$ is not merely a density parameter for the main term, but may act as a \emph{universal amplitude} governing the entire dynamic range of Goldbach fluctuations. If confirmed theoretically, this ``constant consistency principle'' would reveal a deeper organizational structure in prime distribution than previously recognized. These observations motivate future theoretical investigation.

% ============================================================
% REFERENCES
% ============================================================
\bibliographystyle{plainnat}
\begin{thebibliography}{12}

\bibitem[Hardy and Littlewood(1923)]{HL1923}
G.H. Hardy and J.E. Littlewood.
\newblock Some problems of `{P}artitio {N}umerorum' {III}: On the expression of a number as a sum of primes.
\newblock \emph{Acta Mathematica}, 44:1--70, 1923.

\bibitem[Chen(2026a)]{Chen2026a}
R.~Chen.
\newblock Hardy-{L}ittlewood {G}oldbach conjecture validated to $N = 10^{12}$: From transient {U}-distribution to ultimate asymptotic convergence.
\newblock Zenodo, 2026.
\newblock \url{https://doi.org/10.5281/zenodo.18113330}

\bibitem[Chen(2026b)]{Chen2026b}
R.~Chen.
\newblock The crossover phenomenon in {H}ardy-{L}ittlewood {G}oldbach formula: Computational evidence across five orders of magnitude.
\newblock Zenodo, 2026.
\newblock \url{https://doi.org/10.5281/zenodo.18123132}

\bibitem[Chen(2026c)]{Chen2026c}
R.~Chen.
\newblock Spectral rigidity in {G}oldbach representations: Sub-{P}oissonian statistics across thirteen orders of magnitude.
\newblock Zenodo, 2026.
\newblock \url{https://doi.org/10.5281/zenodo.18148544}

\bibitem[Chen(2026d)]{Chen2026d}
R.~Chen.
\newblock The second main term in the asymptotic formula for {G}oldbach representations: {D}irichlet character corrections and their arithmetic origin.
\newblock Zenodo, 2026.
\newblock \url{https://doi.org/10.5281/zenodo.18149305}

\bibitem[Montgomery(1973)]{Montgomery1973}
H.L. Montgomery.
\newblock The pair correlation of zeros of the zeta function.
\newblock \emph{Proceedings of Symposia in Pure Mathematics}, 24:181--193, 1973.

\bibitem[Odlyzko(1987)]{Odlyzko1987}
A.M. Odlyzko.
\newblock On the distribution of spacings between zeros of the zeta function.
\newblock \emph{Mathematics of Computation}, 48(177):273--308, 1987.

\bibitem[Hurst(1951)]{Hurst1951}
H.E. Hurst.
\newblock Long-term storage capacity of reservoirs.
\newblock \emph{Transactions of the American Society of Civil Engineers}, 116:770--808, 1951.

\bibitem[Mandelbrot and Van~Ness(1968)]{Mandelbrot1968}
B.B. Mandelbrot and J.W. Van~Ness.
\newblock Fractional {B}rownian motions, fractional noises and applications.
\newblock \emph{SIAM Review}, 10(4):422--437, 1968.

\bibitem[Goldston et~al.(2009)]{Goldston2009}
D.A. Goldston, J.~Pintz, and C.Y. Y{\i}ld{\i}r{\i}m.
\newblock Primes in tuples {I}.
\newblock \emph{Annals of Mathematics}, 170(2):819--862, 2009.

\bibitem[Maynard(2015)]{Maynard2015}
J.~Maynard.
\newblock Small gaps between primes.
\newblock \emph{Annals of Mathematics}, 181(1):383--413, 2015.

\bibitem[Tao(2023)]{Tao2023}
T.~Tao.
\newblock Correlations of the von {M}angoldt and higher divisor functions.
\newblock \emph{Forum of Mathematics, Pi}, 11:e16, 2023.

\bibitem[Soundararajan(2022)]{Sound2022}
K.~Soundararajan.
\newblock The distribution of prime numbers.
\newblock \emph{Proceedings of the ICM 2022}, 2:1--25, 2022.

\bibitem[Iwaniec and Kowalski(2004)]{IK2004}
H.~Iwaniec and E.~Kowalski.
\newblock \emph{Analytic Number Theory}.
\newblock American Mathematical Society, 2004.

\bibitem[Davenport(2000)]{Davenport2000}
H.~Davenport.
\newblock \emph{Multiplicative Number Theory}.
\newblock Springer, 3rd edition, 2000.

\bibitem[Languasco and Zaccagnini(2012)]{LZ2012}
A.~Languasco and A.~Zaccagnini.
\newblock On the constant in the {M}ertens product for arithmetic progressions.
\newblock \emph{Mathematics of Computation}, 81(279):1813--1829, 2012.

\end{thebibliography}

% ============================================================
% APPENDIX
% ============================================================
\appendix

\section{FFT Algorithm}
\label{app:algorithm}

\begin{algorithm}[H]
\caption{Riemann Radar: L-Function Zero Detection}
\begin{algorithmic}[1]
\REQUIRE Goldbach data $\{(N_i, c_p(N_i))\}$ for even $N_i \in [N_{\min}, N_{\max}]$
\ENSURE Power spectrum $P(\gamma)$ with peaks at L-function zeros

\STATE \textbf{Transform to log domain:} $t_i \leftarrow \ln(N_i)$
\STATE \textbf{Set grid:} $\Delta t \leftarrow 0.01$, $M \leftarrow \lfloor(t_{\max} - t_{\min})/\Delta t\rfloor$
\STATE \textbf{Interpolate:} Cubic spline to uniform grid $\{t_k = t_{\min} + k\Delta t\}_{k=0}^{M-1}$
\STATE \textbf{Detrend:} $r_k \leftarrow c_p(t_k) - C_2/(2t_k)$; then linear detrend
\STATE \textbf{Window:} $r_k \leftarrow r_k \cdot \frac{1}{2}(1 - \cos(2\pi k/(M-1)))$ \COMMENT{Hanning}
\STATE \textbf{FFT:} $\hat{R} \leftarrow \text{FFT}(\{r_k\})$
\STATE \textbf{Power:} $P_j \leftarrow |\hat{R}_j|^2$
\STATE \textbf{Calibrate:} $\gamma_j \leftarrow 2\pi j/(M \cdot \Delta t)$
\RETURN $\{(\gamma_j, P_j)\}$
\end{algorithmic}
\end{algorithm}

\section{Numerical Parameters}
\label{app:params}

\begin{table}[H]
\centering
\caption{Summary of key quantities}
\begin{tabular}{@{}llll@{}}
\toprule
\textbf{Quantity} & \textbf{Value} & \textbf{Error} & \textbf{Description} \\
\midrule
$C_2$ & $0.6601618$ & exact & Twin prime constant \\
$\kappa$ & $0.70$ & $\pm 0.04$ & Universal amplitude \\
$\kappa/C_2$ & $1.06$ & $\pm 0.06$ & Ratio (conjectured = 1) \\
$H$ (detrended) & $0.85$ & $\pm 0.02$ & Hurst exponent \\
$H$ (increments) & $0.40$ & $\pm 0.02$ & Increment persistence \\
$H$ (null) & $0.51$ & $\pm 0.03$ & Control baseline \\
$\Delta\gamma$ & $1.25$ & --- & FFT resolution \\
$N_{\max}$ & $1.5 \times 10^6$ & --- & Data range \\
\bottomrule
\end{tabular}
\end{table}

\section{Sampling Density}
\label{app:sampling}

For the main analysis:
\begin{itemize}
\item Original data: $\sim 7.5 \times 10^5$ even integers in $[10^4, 1.5 \times 10^6]$
\item Log-domain range: $t \in [9.21, 14.22]$, span $= 5.01$
\item Interpolated grid: $M = 501$ points at $\Delta t = 0.01$
\item Nyquist frequency: $\gamma_{\text{Nyquist}} = \pi/\Delta t = 314$
\item Hanning window: $w(k) = 0.5(1 - \cos(2\pi k/(M-1)))$, $k = 0, \ldots, M-1$
\item Equivalent noise bandwidth: $\text{ENBW} = 1.5$ bins
\end{itemize}

The sampling is well above Nyquist for all L-function zeros of interest ($\gamma < 50$).

\section{Reproducibility and Data Availability}

All computations were performed using:
\begin{itemize}
\item Python 3.12 with NumPy 1.26, SciPy 1.11
\item Cubic spline interpolation: \texttt{scipy.interpolate.CubicSpline}
\item FFT: \texttt{numpy.fft.fft}
\item R/S analysis: Custom implementation following \citet{Hurst1951}
\end{itemize}

The algorithms for the ``Riemann Radar'' FFT analysis and the Hurst exponent calculation are detailed in Appendix~\ref{app:algorithm} to ensure full reproducibility. 

\textbf{Code and Data Availability:} All source code, data, and supplementary materials are publicly available at:
\begin{center}
\url{https://github.com/Ruqing1963/goldbach-fractal-analysis}
\end{center}

\end{document}
